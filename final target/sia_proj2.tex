\documentclass[11pt]{article}

\usepackage[portuguese]{babel}
\usepackage[utf8]{inputenc}
\usepackage{amsmath}
\usepackage{graphicx}
\usepackage{float}
\usepackage{subfig}
\usepackage{fixltx2e}
\usepackage[bottom]{footmisc}
\usepackage{color}
\usepackage{xargs}                      % Use more than one optional parameter in a new commands
\usepackage[pdftex,dvipsnames]{xcolor}  % Coloured text etc.
\usepackage[colorinlistoftodos,prependcaption,textsize=tiny]{todonotes}
\newcommandx{\unsure}[2][1=]{\todo[linecolor=red,backgroundcolor=red!25,bordercolor=red,#1]{#2}}
\newcommandx{\change}[2][1=]{\todo[linecolor=blue,backgroundcolor=blue!25,bordercolor=blue,#1]{#2}}
\newcommandx{\info}[2][1=]{\todo[linecolor=OliveGreen,backgroundcolor=OliveGreen!25,bordercolor=OliveGreen,#1]{#2}}
\newcommandx{\improvement}[2][1=]{\todo[linecolor=Plum,backgroundcolor=Plum!25,bordercolor=Plum,#1]{#2}}
\newcommandx{\thiswillnotshow}[2][1=]{\todo[disable,#1]{#2}}
\usepackage[font=footnotesize]{caption}

\numberwithin{equation}{section}

\linespread{1.3}
\usepackage{indentfirst}
\usepackage[top=2cm, bottom=2cm, right=2.25cm, left=2.25cm]{geometry}
\addto\captionsportuguese{\renewcommand{\contentsname}{Índice}}

\begin{document}

\begin{titlepage}
\begin{center}

\hfill \break
\hfill \break

\includegraphics[width=0.3\textwidth]{./logo}~\\[1cm]

\textsc{\LARGE Instituto Superior Técnico}\\[0.25cm]
\textsc{\Large Mestrado Integrado em Engenharia Electrotécnica e de Computadores}\\[1.8cm]
\textsc{\huge Sistemas Integrados Analógicos}\\[0.25cm]

{\huge \bfseries \textit{Design} de um Amplificador \\[1cm]}

\begin{tabular}{ l l }
João Bernardo Sequeira de Sá & \hspace{2mm} n.º 68254 \\
Maria Margarida Dias dos Reis & \hspace{2mm} n.º 73099 \\
Nuno Miguel Rodrigues Machado & \hspace{2mm} n.º 74236
\end{tabular}

\vfill

{\large Lisboa, 31 de Maio de 2015} 

\end{center}
\end{titlepage}

\pagenumbering{gobble}
\clearpage

\tableofcontents
\pagebreak

\clearpage
\pagenumbering{arabic}

\section{Introdução}

Pretende-se projectar um amplificador \textit{folded cascode} CMOS OTA de dois andares de acordo com as especificações da seguinte tabela.

\begin{table}[H]
	\centering
	\caption{Características do amplificador a projectar.}
	\vspace{-1.5mm}
	\includegraphics[keepaspectratio=true, scale=0.45]{teoricas/tabela1}
\end{table}

O circuito de ponto de partida para a realização do projecto é apresentado de seguida.

\begin{figure}[H]
	\centering
	\includegraphics[keepaspectratio=true, scale=0.50]{teoricas/circuito1}
	\vspace{-0.5em}
	\caption{Circuito do amplificador a projectar.}
	\vspace{-0.8em}
\end{figure} 

\section{Adenda ao \textit{Middle Target}}

\section{Projecção do \textit{Layout}}

\section{}

\pagebreak

\section{Conclusões}

\end{document}