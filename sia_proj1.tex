\documentclass[11pt]{article}

\usepackage[portuguese]{babel}
\usepackage[utf8]{inputenc}
\usepackage{amsmath}
\usepackage{graphicx}
\usepackage{float}
\usepackage{subfig}
\usepackage{fixltx2e}
\usepackage[bottom]{footmisc}
\usepackage{color}
\usepackage[usenames,dvipsnames]{xcolor}
\usepackage[font=footnotesize]{caption}

\numberwithin{equation}{section}

\linespread{1.3}
\usepackage{indentfirst}
\usepackage[top=2cm, bottom=2cm, right=2.5cm, left=2.5cm]{geometry}
\addto\captionsportuguese{\renewcommand{\contentsname}{Índice}}

\begin{document}

\begin{titlepage}
\begin{center}

\hfill \break
\hfill \break

\includegraphics[width=0.3\textwidth]{./logo}~\\[1cm]

\textsc{\LARGE Instituto Superior Técnico}\\[0.25cm]
\textsc{\Large Mestrado Integrado em Engenharia Electrotécnica e de Computadores}\\[1.8cm]
\textsc{\huge Sistemas Integrados Analógicos}\\[0.25cm]

{\huge \bfseries Projecto de Alto Nível de um ADC e DAC \\[1cm]}

\begin{tabular}{ l l }
Maria Margarida Dias dos Reis & \hspace{2mm} n.º 73099 \\
Nuno Miguel Rodrigues Machado & \hspace{2mm} n.º 74236

\end{tabular}

\vfill

{\large Lisboa, 16 de Março de 2015} 

\end{center}
\end{titlepage}

\pagenumbering{gobble}
\clearpage

\tableofcontents
\pagebreak

\clearpage
\pagenumbering{arabic}

\section{Introdução}

Com este trabalho laboratorial pretende-se introduzir o \textit{software} Cadence, projectando um conversor AD/DA de alto nível. Analisando os conversores analógico-digitais (ADC) pode-se melhor compreender o conceito de \textit{Fast Fourier Transform} (FFT), e a maneira como pode ser aplicada para medir parâmetros dos ADC, como a SINAD e o ENOB. Pretende-se também estudar o efeito de aplicar diversas janelas sobre a FFT.

\section{Introdução Teórica}
\subsection{Conversores A/D e Conversores D/A}

Começando por analisar os conversores analógico-digitais, as arquitecturas que os permitem podem ser divididas em três categorias: baixa-a-média velocidade, média velocidade e alta velocidade. O ADC utilizado neste trabalho é de aproximações sucessivas (SAR), sendo de média velocidade e exactidão. 

Os conversores deste tipo estão entre os mais populares para realizar ADCs devido à sua versatilidade - conseguem efectuar conversões rápidas ou podem ser utilizados para que haja uma maior exactidão, operando a baixa potência nos dois casos. Este conjunto de características deriva de, no caso mais simples, o conversor necessitar apenas de um só comparador, um banco de condensadores com interruptores e pouca lógica de controlo digital. Na figura abaixo está esquematizado o circuito referido.

\begin{figure}[h]
	\centering
	\includegraphics[keepaspectratio=true, scale=0.45]{./teoricas/SAR_1}
	\caption{ADC construído com uma arquitectura de aproximações sucessivas.}
	\vspace{-0.8em}
\end{figure}

OS ADCs de aproximações sucessivas têm por base o algoritmo de procura conhecido como ``procura binária'', onde os dados podem ser calculados em $N$ passos, para um conjunto de dados organizados de tamanho $2^N$.

Este algoritmo EXPLICAR AQUI

Assim, o conversor aplica o algoritmo para determinar a palavra digital mais próxima que corresponde ao sinal de entrada. Isto implica que são necessários $N$ ciclos de relógio para completar uma conversão de $N$ \textit{bits}.

O diagrama de blocos de um ADC unipolar de aproximações sucessivas que utiliza também um DAC é apresentado de seguida.
	
\pagebreak	

\begin{figure}[h]
	\centering
	\includegraphics[keepaspectratio=true, scale=0.30]{./teoricas/SAR_2}
	\caption{Diagrama de blocos de um ADC de aproximações sucessivas.}
	\vspace{-0.8em}
\end{figure}

Existe um circuito \textit{sample-and-hold} que permite adquirir a tensão de entrada. De seguida um comparador analógico de tensão compara a tensão de entrada com a saída do DAC e coloca o resultado da comparação no registo de aproximações sucessivas (SAR). O SAR fornece ao DAC um código digital da tensão de entrada e então, o DAC, para comparação com a tensão de referência, fornece ao comparador uma tensão analógica igual ao código digital que saiu do SAR. 

\subsection{SINAD, SNR e ENOB}

O \textit{signal-to-noise and distortion} (SINAD) é uma medida da \textit{perfomance} dinâmica geral de um ADC. É o rácio entre a amplitude do sinal em \textit{root-mean-square} (valor eficaz) e o valor médio da \textit{root-sum-square} das restantes componentes espectrais, incluindo harmónicas, mas excluindo a componente DC. O cálculo do SINAD em $dB$ é feito de acordo com a equação~\ref{eq:SINAD}.

\vspace{-3mm}
\begin{equation}
\text{SINAD} = 10\times \log_{10} \left(\frac{A^{2}_{bin(f_{in})}}{\sum_{n=2}^{\text{size}/2}\left(A^{2}_{n} - A^{2}_{bin(f_{in})}\right)}\right)
\label{eq:SINAD}
\end{equation}

\vspace{1mm}
O \textit{signal-to-noise ratio} (SNR) é calculado a partir dos dados da FFT, tal como a SINAD, mas as harmónicas do sinal são excluídas dos cálculos, deixando apenas os termos de ruído. De uma maneira mais abstracta pode ser descrito como a comparação entre o nível de sinal desejado ao nível de ruído. O cálculo do SNR em $dB$ é feito de acordo com a equação~\ref{eq:SNR}.

\vspace{-3mm}
\begin{equation}
\text{SNR} = 6.02N + 1.76
\label{eq:SNR}
\end{equation}

\vspace{1mm}
O \textit{effective number of bits} (ENOB) é uma medida da resolução de um ADC. De facto, a resolução de um ADC é dada pelo número de \textit{bits} que são utilizados para representar um valor analógico porém, todos os ADCs reais introduzem ruído e distorção. Assim, o ENOB especifica o número efectivo de \textit{bits} que se tem na realidade, quando se considera a existência de ruído. O cálculo do ENOB é feito de acordo com a equação~\ref{eq:ENOB}.

\vspace{-3mm}
\begin{equation}
\text{ENOB} = \frac{\text{SINAD}-1.76}{6.02}
\label{eq:ENOB}
\end{equation}

\subsection{Janela Rectangular, Janela de Hamming e Janela de Blackman-Harris}

Sobre a FFT podem-se aplicar diversas janelas, procurando reduzir assim 

\end{document}