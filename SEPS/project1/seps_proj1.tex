% !TeX spellcheck = pt_BR
\documentclass[11pt]{article}

\usepackage[portuguese]{babel}
\usepackage[utf8]{inputenc}
\usepackage{amsmath}
\usepackage{graphicx}
\usepackage{float}
\usepackage{subfig}
\usepackage{fixltx2e}
\usepackage[bottom]{footmisc}
\usepackage{listings}
\usepackage{color} 
\usepackage[usenames,dvipsnames]{xcolor}
\usepackage[colorinlistoftodos]{todonotes}
\usepackage[font=footnotesize]{caption}

\numberwithin{equation}{section}

\linespread{1.3}
\usepackage{indentfirst}
\usepackage[top=2cm, bottom=2cm, right=2.5cm, left=2.5cm]{geometry}
\addto\captionsportuguese{\renewcommand{\contentsname}{Índice}}

\begin{document}

\begin{titlepage}
\begin{center}

\hfill \break
\hfill \break

\includegraphics[width=0.3\textwidth]{./logo}~\\[1cm]

\textsc{\LARGE Instituto Superior Técnico}\\[0.25cm]
\textsc{\Large Mestrado Integrado em Engenharia Electrotécnica e de Computadores}\\[1.8cm]
\textsc{\huge Sistemas Electrónicos de Processamento de Sinal}\\[0.25cm]

\vspace{6mm}

{\huge \bfseries Introdução ao Processamento Digital de Sinal \& \linebreak BPSK Modem \\[1cm]}

\begin{tabular}{ l l }
Maria Margarida Dias dos Reis & \hspace{2mm} n.º 73099 \\
David Gonçalo C. C. de Deus Oliveira & \hspace{2mm} n.º 73722 \\
Nuno Miguel Rodrigues Machado & \hspace{2mm} n.º 74236

\end{tabular}

\vfill

{\large Lisboa, 16 de Março de 2015} 

\end{center}
\end{titlepage}

\pagenumbering{gobble}
\clearpage

\tableofcontents
\pagebreak

\clearpage
\pagenumbering{arabic}

\section{Introdução}

\section{Projecto $\#$1 - NCO}

\todo{intro - margarida}

\subsection{} %pergunta 1

\todo{margarida}

\subsection{} %pergunta 2

\todo{david}

\subsection{} %pergunta 3

\todo{david}

\subsection{} %pergunta 4

Foram criadas duas variáveis com o objectivo de controlar a amplitude e frequência do sinal sinusoidal. A variável \texttt{delta} representa o controlo da frequência, já a variável \texttt{amp} representa o controlo da amplitude. Como se pode identificar no código abaixo:  

\begin{lstlisting}[language=C]
void main(){
	 ....
	 /* Variavel de controlo de frequencia */
	 short 	delta = 0 ;
	 //Variavel de amplitude: Define um ganho de 1/2 
	 short	amp = 16384; 
	 ...
	 while(1){                	   	//infinite loop
		  if(intflag != FALSE){
		  ...	
		  //Obtencao do valor para a frequencia		
		  delta = 16384 + (inbuf>>2); 
		  ....
		  //Controlo da amplitude e frequencia
		  y = (amp*((y1*delta)>> 15)>>15);
		
		  if(status < 0)
			y = -y;
			
		  AIC_buffer.channel[LEFT] = y;
		  }
	 }
}
\end{lstlisting}

Como foi referido anteriormente na questão 2.1, a partir da rampa de integração pode-se calcular o valor de \texttt{delta} usando a seguinte relação:

\vspace{-3mm}
\begin{equation}
\bigtriangleup = \frac{f_{0}}{f_{s}} 2^{16},
\label{eq:freq_bin}
\end{equation}

Como o NCO tem como característica uma a frequência $f_{0}$ que varia entre $2$kHz e $6$kHz. Estes valores são controlados a partir da amplitude do sinal de entrada. Quando esta for minima, a frequência $f_{0}$ é de $2$kHz e quando for máxima, a frequência $f_{0}$ é de $6$kHz. Com estas especificações pode-se calcular o valor de \texttt{delta}, todos os valores referidos na tabela seguinte estão no formato $Q_{15}$:

\begin{table}[h]
	\centering
	\begin{tabular}{|c|c|}
		\hline
	\underline{$f_{0}$ kHz} &\underline{\texttt{delta}} \\ \hline
		$2$ & $8192$  \\ \hline
		$4$ & $16384$ \\ \hline
		$6$ & $24576$ \\ \hline
	\end{tabular}
\end{table}

Analisando este resultado verificou-se que o valor de \texttt{delta} oscila com uma amplitude de $8192$ em torno de $16384$. Ou seja, $f_{0}$ têm uma frequência central em 4 kHz oscilando com uma amplitude de 2 kHz. Com esta conclusão, teve-se de garantir que o valor da amplitude do sinal de entrada não ultrapasse os $8192$, mantendo a relação entre cada amostra. Optou-se por dividir o valor de cada amostra por $4$, deslocamento de 2 bits para a direita, pois a amplitude máxima é de $32768$. Em baixo está o código referente ao calculo  para obter o delta \texttt{delta}, todos as variáveis definidas neste excerto são de 16 bits, \texttt{short}, em formato $Q_{15}$:

\begin{lstlisting}[language=C]
		...	
		//Obtencao do valor para a frequencia		
		delta = 16384 + (inbuf>>2); 
		....
\end{lstlisting}



\subsection{} %pergunta 5

\todo{teddy}

\subsection{} %pergunta 6

\todo{teddy}

\subsection{} %pergunta 7

\todo{margarida}

\subsection{} %pergunta 8

\todo{margarida}

\section{Projecto $\#$2 - Transmissor BPSK}

\todo{intro - david}

\subsection{} %pergunta 1

\todo{david}

\subsection{} %pergunta 2

\todo{teddy}

\subsection{} %pergunta 3

\todo{margarida}

\section{Conclusões}

\end{document}