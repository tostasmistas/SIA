\documentclass[11pt]{article}

\usepackage[portuguese]{babel}
\usepackage[utf8]{inputenc}
\usepackage{amsmath}
\usepackage{graphicx}
\usepackage{float}
\usepackage{subfig}
\usepackage{fixltx2e}
\usepackage[bottom]{footmisc}
\usepackage{color}
\usepackage[usenames,dvipsnames]{xcolor}
\usepackage[colorinlistoftodos]{todonotes}
\usepackage[font=footnotesize]{caption}

\numberwithin{equation}{section}

\linespread{1.3}
\usepackage{indentfirst}
\usepackage[top=2cm, bottom=2cm, right=2.25cm, left=2.25cm]{geometry}
\addto\captionsportuguese{\renewcommand{\contentsname}{Índice}}

\begin{document}

\begin{titlepage}
\begin{center}

\hfill \break
\hfill \break

\includegraphics[width=0.3\textwidth]{./logo}~\\[1cm]

\textsc{\LARGE Instituto Superior Técnico}\\[0.25cm]
\textsc{\Large Mestrado Integrado em Engenharia Electrotécnica e de Computadores}\\[1.8cm]
\textsc{\huge Sistemas Integrados Analógicos}\\[0.25cm]

{\huge \bfseries \textit{Design} de um Amplificador e ADC de 4 \textit{bits} \\[1cm]}

\begin{tabular}{ l l }
João Bernardo Sequeira de Sá & \hspace{2mm} n.º 68254 \\
Maria Margarida Dias dos Reis & \hspace{2mm} n.º 73099 \\
Nuno Miguel Rodrigues Machado & \hspace{2mm} n.º 74236
\end{tabular}

\vfill

{\large Lisboa, 1 de Maio de 2015} 

\end{center}
\end{titlepage}

\pagenumbering{gobble}
\clearpage

\tableofcontents
\pagebreak

\clearpage
\pagenumbering{arabic}

\section{Introdução}

Pretende-se projectar um amplificador \textit{folded cascode} CMOS OTA de dois andares de acordo com as especificações da seguinte tabela.

\begin{table}[H]
	\centering
	\caption{Características do amplificador a projectar.}
	\vspace{-1.5mm}
	\includegraphics[keepaspectratio=true, scale=0.45]{teoricas/tabela1}
\end{table}

O circuito de ponto de partida para a realização do projecto é apresentado de seguida.

\begin{figure}[H]
	\centering
	\includegraphics[keepaspectratio=true, scale=0.50]{teoricas/circuito1}
	\vspace{-0.5em}
	\caption{Circuito do amplificador a projectar.}
	\vspace{-0.8em}
\end{figure} 

\pagebreak

\section{Abordagem ao Circuito}

\subsection{Funcionamento Teórico do Circuito}

\todo{introducao teorica do OTA}

Analisando o circuito da Figura 1 em pormenor identificam-se 5 blocos, sendo importante analisar a função de cada um, para que melhor se possa compreender o funcionamento e comportamento do circuito na sua totalidade. 

O Bloco 1 representa o transístor responsável pela polarização do circuito. O Bloco 2 representa um par diferencial PMOS. O Bloco 3 corresponde a um espelho de corrente \textit{cascode} básico do tipo PMOS. O Bloco 4 actua como isolamento. O Bloco 5 funciona como fonte de corrente que ``puxa'' sempre $I$ (corrente de M\textsubscript{11}) para o \textit{ground}.

Relativamente ao par diferencial, o circuito pode funcionar de acordo com três situações:

\begin{itemize}
	\vspace{-3mm}
	\item $v_{in-} = v_{in+} \rightarrow$ situação 1
	\vspace{-1.5mm}
	\item $v_{in-} > v_{in+} \rightarrow$ situação 2
	\vspace{-1.5mm}
	\item $v_{in-} < v_{in+} \rightarrow$ situação 3
\end{itemize}

Na situação 1, cada transístor do par diferencial, M\textsubscript{1} e M\textsubscript{2}, tem metade da corrente que passa em M\textsubscript{11} e o circuito apresenta o seguinte comportamento.

\begin{figure}[H]
	\centering
	\includegraphics[keepaspectratio=true, scale=0.50]{teoricas/situacao1}
	\vspace{-0.5em}
	\caption{Funcionamento do circuito na situação 1.}
	\vspace{-0.8em}
\end{figure} 

Considerando agora o extremo da situação 2, a tensão na \textit{gate} de M\textsubscript{1} toma o valor máximo da fonte de tensão que polariza esse transístor e a tensão na \textit{gate} de M\textsubscript{2} é nula. Assim, o circuito apresenta o seguinte comportamento.

\begin{figure}[H]
	\centering
	\includegraphics[keepaspectratio=true, scale=0.50]{teoricas/situacao2}
	\vspace{-0.5em}
	\caption{Funcionamento do circuito no extremo da situação 2.}
	\vspace{-0.8em}
\end{figure} 

Considerando agora o extremo da situação 3, a tensão na \textit{gate} de M\textsubscript{2} toma o valor máximo da fonte de tensão que polariza esse transístor e a tensão na \textit{gate} de M\textsubscript{1} é nula. Assim, o circuito apresenta o seguinte comportamento.

\begin{figure}[H]
	\centering
	\includegraphics[keepaspectratio=true, scale=0.50]{teoricas/situacao3}
	\vspace{-0.5em}
	\caption{Funcionamento do circuito no extremo da situação 3.}
	\vspace{-0.8em}
\end{figure} 

\subsection{Primeiro Dimensionamento dos Transístores}

A primeira fase no projecto do amplificador passou por decidir as dimensões dos vários transístores. Sabe-se que a dimensão de um transístor é dada pelos parâmetros $W$ (\textit{width} - largura) e $L$ (\textit{lenght} - comprimento). 

O valor de $L$ ficou decidido à partida como sendo 1 $\mu$m para todos os transístores do circuito, isto porque se tem como \textit{rule of thumb} que, para se evitar o efeito de modulação do comprimento do canal, o valor de $L$ deve ser maior ou igual a 1 $\mu$m. O valor de $W$ pode ser calculado recorrendo à equação que determina a corrente num transístor. Para um transístor do tipo P a corrente é dada por

\vspace{-3mm}
\begin{equation}
I_{D} = \frac{1}{2}\mu_{n}C_{ox}\times \Big(\frac{W}{L}\Big) \times(V_{GS}-V_{TH})^2 = k_P \times \Big(\frac{W}{L}\Big) \times V_{OD}^2,
\label{eq:corrente}
\end{equation}

\vspace{1mm}
sendo que para um transístor do tipo N troca o valor do factor de ganho, em vez de $k_P$ tem-se $k_N$.

Da equação anterior pretende-se determinar o valor de $W$ dos vários transístores, sendo então necessário saber o valor de $L$ (já determinado anteriormente), o valor da corrente que passa nos transístores, $I_{D}$, o valor de $k$ e o valor da tensão de \textit{overdrive}, $V_{OD}$.

O valor da tensão de \textit{overdrive} definiu-se como sendo de 0.2 V para todos os transístores. Este valor deriva de outra \textit{rule of thumb} que indica que se deve escolher para $V_{OD}$ um valor de 0.2V - menos do que isso e fica-se demasiado sensível a $V_{TH}$ e mais do que isso e fica-se com pouca margem de saturação, que é uma medida do quão dentro da saturação se está, sendo calculada por $V_{DS} - V_{OD}$.

O valor de $k$ pode ser obtido com recurso aos \textit{process parameters}, sendo de referir que o valores que se retiram das \textit{datasheets} representam apenas $\mu_{n}C_{ox}$, pelo que têm de ser multiplicados por $1/2$ para que se obtenha o factor de ganho final, como se pode ver na próxima equação, para o caso de um transístor do tipo P:

\vspace{-3mm}
\begin{equation}
k_P = \frac{1}{2}\mu_{n}C_{ox} = \frac{1}{2} \times KP_P.
\end{equation}

\vspace{1mm}
Os valores já conhecidos que ajudam a obter o valor de $W$ através da equação (2.1) encontram-se esquematizados na seguinte tabela.

\begin{table}[H]
	\centering
	\caption{Valores especificados para algumas das características que definem os transístores.}
	\vspace{-1.5mm}
	\includegraphics[keepaspectratio=true, scale=0.45]{teoricas/tabela2}
\end{table}

Para determinar os valores das correntes que passam nos vários transístores começou-se por determinar a corrente máxima à saída do circuito. Existe uma relação entre a \textit{slew-rate}, SR, e a corrente de saída máxima, $I_{out_{max}}$ expressa por

\vspace{-3mm}
\begin{equation}
\text{SR} = \frac{I_{out_{max}}}{C_L},
\end{equation}

\vspace{1mm}
que nos permite concluir que quanto maior for a corrente de saída, mais depressa é carregado o condensador que constitui a carga.

Com os valores da Tabela 1 obtém-se:

\vspace{-3mm}
\begin{equation}
\text{SR} = \frac{I_{out_{max}}}{C_L} \leftrightarrow I_{out_{max}} = 200 \times 0.25 \times 10^{-6}~\text{A} = 50~\mu \text{A}.
\end{equation}

\vspace{1mm}
Analisando as Figuras 3 a 4 percebe-se que a corrente $I_{out_{max}}$ corresponde a $I/2$, pelo que o valor máximo de $I$ corresponde a 100 $\mu$A. O dimensionamento dos transístores foi feito tendo em conta o máximo de corrente que por eles passa, de acordo com

\vspace{-3mm}
\begin{equation}
W_P  = \frac{I_{D_{max}} \times L}{k_P \times V_{OD}^2} \rightarrow \text{transístor tipo PMOS};
\end{equation}
\vspace{1mm}
\begin{equation}
W_N  = \frac{I_{D_{max}} \times L}{k_N \times V_{OD}^2} \rightarrow \text{transístor tipo NMOS}.
\end{equation}

\vspace{1mm}
Os valores de $W$, calculados em função da corrente máxima que atravessa os vários transístores, apresentam-se na próxima tabela. De notar que os valores foram arredondados ao inteiro mais próximo.

\begin{table}[H]
	\centering
	\caption{Valores de $W$ dos transístores que constituem o circuito, calculados em função da corrente máxima.}
	\vspace{-1.5mm}
	\includegraphics[keepaspectratio=true, scale=0.45]{teoricas/valoresW}
\end{table}

De notar que os transístores M\textsubscript{5} e M\textsubscript{6} têm as mesmas dimensões, tal como pretendido, pois formam um espelho de corrente que tem como rácio 1:1. O mesmo se aplica aos transístores M\textsubscript{7} e M\textsubscript{8}.

\subsection{Primeira Simulação}

Com o dimensionamento dos transístores feito procede-se a uma primeira simulação do circuito, com o intuito de verificar o seu funcionamento. Porém, antes de simular o circuito alterou-se a sua polarização, para que em vez de ser feita em tensão seja feita em corrente. Isto é feito porque uma polarização em corrente permite ter mais controlo, sendo que quando é feita em tensão não se tem garantias dos valores pretendidos.  

Assim, o circuito da Figura 1 foi alterado para o apresentado de seguida.

\todo{imagem teorica circuito alterado}

Como se pode ver, o transístor M\textsubscript{11} que é originalmente polarizado em tensão com $V_{BIAS}$, Bloco 1, foi substituído por um espelho de corrente básico que é polarizado em corrente com $I_{BIAS}$. A polarização feita com recurso a $V_{BIAS_{2}}$ e $V_{BIAS_{3}}$ também foi alterada para passar a ser feita em corrente.

O \textit{schematic} criado no Cadence que permite simular o circuito é apresentado de seguida.

\todo{imagem do schematic}

\todo{simulacao e verificacao do vgs}


0 - corte
1 - triodo
2 - saturacao
3- subthreshold

No entanto, concluiu-se que o dimensionamento feito em função do máximo da corrente que percorre cada transístor não é a melhor opção pois a tensão $V_{GS}$

\begin{table}[H]
	\centering
	\caption{Valores de $W$ dos transístores que constituem o circuito, calculados em função do PFR.}
	\vspace{-1.5mm}
	\includegraphics[keepaspectratio=true, scale=0.45]{teoricas/valoresW2}
\end{table}

\pagebreak

\section{Conclusões}

\end{document}