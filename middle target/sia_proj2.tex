\documentclass[11pt]{article}

\usepackage[portuguese]{babel}
\usepackage[utf8]{inputenc}
\usepackage{amsmath}
\usepackage{graphicx}
\usepackage{float}
\usepackage{subfig}
\usepackage{fixltx2e}
\usepackage[bottom]{footmisc}
\usepackage{color}
\usepackage[usenames,dvipsnames]{xcolor}
\usepackage[colorinlistoftodos]{todonotes}
\usepackage[font=footnotesize]{caption}

\numberwithin{equation}{section}

\linespread{1.3}
\usepackage{indentfirst}
\usepackage[top=2cm, bottom=2cm, right=2.25cm, left=2.25cm]{geometry}
\addto\captionsportuguese{\renewcommand{\contentsname}{Índice}}

\begin{document}

\begin{titlepage}
\begin{center}

\hfill \break
\hfill \break

\includegraphics[width=0.3\textwidth]{./logo}~\\[1cm]

\textsc{\LARGE Instituto Superior Técnico}\\[0.25cm]
\textsc{\Large Mestrado Integrado em Engenharia Electrotécnica e de Computadores}\\[1.8cm]
\textsc{\huge Sistemas Integrados Analógicos}\\[0.25cm]

{\huge \bfseries \textit{Design} de um Amplificador e ADC de 4 \textit{bits} \\[1cm]}

\begin{tabular}{ l l }
João Bernardo Sequeira de Sá & \hspace{2mm} n.º 68254 \\
Maria Margarida Dias dos Reis & \hspace{2mm} n.º 73099 \\
Nuno Miguel Rodrigues Machado & \hspace{2mm} n.º 74236
\end{tabular}

\vfill

{\large Lisboa, 1 de Maio de 2015} 

\end{center}
\end{titlepage}

\pagenumbering{gobble}
\clearpage

\tableofcontents
\pagebreak

\clearpage
\pagenumbering{arabic}

\section{Introdução}

Pretende-se projectar um amplificador \textit{folded cascode} CMOS OTA de dois andares de acordo com as especificações da seguinte tabela.

\begin{table}[H]
	\centering
	\caption{Características do amplificador a projectar.}
	\vspace{-1.5mm}
	\includegraphics[keepaspectratio=true, scale=0.45]{teoricas/tabela1}
\end{table}

O circuito de ponto de partida para a realização do projecto é apresentado de seguida.

\begin{figure}[H]
	\centering
	\includegraphics[keepaspectratio=true, scale=0.50]{teoricas/circuito1}
	\vspace{-0.5em}
	\caption{Circuito do amplificador a projectar.}
	\vspace{-0.8em}
\end{figure} 

\section{Abordagem do Circuito}

\subsection{Identificação dos Blocos do Circuito}

Analisando o circuito da Figura 1 em pormenor identificam-se 5 blocos, sendo importante analisar a função de cada um, para que melhor se possa compreender o funcionamento e comportamento do circuito na sua totalidade. 

O Bloco 1 representa o transístor responsável pela polarização do circuito. O Bloco 2 representa um par diferencial PMOS. O Bloco 3 corresponde a um espelho de corrente \textit{cascode} básico do tipo PMOS.

\todo{blocos 4 e 5?}

\subsection{Definicação das Dimensões dos Transístores}

A primeira fase no projecto do amplificador passou por decidir as dimensões dos vários transístores. Sabe-se que a dimensão de um transístor é dada pelos parâmetros $W$ (\textit{width} - largura) e $L$ (\textit{lenght} - comprimento). 

O valor de $L$ ficou decidido à partida como sendo 1 $\mu$m para todos os transístores do circuito, isto porque se tem como \textit{rule of thumb} que, para se evitar o efeito de modulação do comprimento do canal, o valor de $L$ deve ser maior ou igual a 1 $\mu$m. O valor de $W$ pode ser calculado recorrendo à equação que determina a corrente num transístor:

\vspace{-3mm}
\begin{equation}
I_{D} = \frac{1}{2}\mu_{n}C_{ox}\times \Big(\frac{W}{L}\Big) \times(V_{GS}-V_{TH})^2 = k\times \Big(\frac{W}{L}\Big) \times V_{OD}^2.
\label{eq:corrente}
\end{equation}

\vspace{1mm}
Da equação anterior pretende-se determinar o valor de $W$ dos vários transístores, sendo então necessário saber o valor de $L$ (já determinado anteriormente), o valor da corrente que passa nos transístores, $I_{D}$, o valor de $k$ e o valor da tensão de \textit{overdrive}, $V_{OD}$.

O valor da tensão de \textit{overdrive} definiu-se como sendo de 0.2 V para todos os transístores. Este valor deriva de outra \textit{rule of thumb} que indica que se deve escolher para $V_{OD}$ um valor de 0.2V - menos do que isso e fica-se demasiado sensível a $V_{TH}$ e mais do que isso e fica-se com pouca margem de saturação, que é uma medida do quão dentro da saturação se está, sendo calculada por $V_{DS} - V_{OD}$.

O valor de $k$ pode ser obtido com recurso aos \textit{process parameters}, sendo de referir que o valores que se retiram das \textit{datasheets} representam apenas $\mu_{n}C_{ox}$, pelo que têm de ser multiplicados por $1/2$ para que se obtenha o factor de ganho final, como se pode ver na próxima equação, para o caso de um transístor do tipo P:

\vspace{-3mm}
\begin{equation}
k_P = \frac{1}{2}\mu_{n}C_{ox} = \frac{1}{2} \times KP_P.
\end{equation}

\vspace{1mm}
Os valores já conhecidos que ajudam a obter o valor de $W$ através da equação (2.1) encontram-se esquematizados na seguinte tabela.

\begin{table}[H]
	\centering
	\caption{Valores especificados para algumas das características que definem os transístores.}
	\vspace{-1.5mm}
	\includegraphics[keepaspectratio=true, scale=0.45]{teoricas/tabela2}
\end{table}

\end{document}